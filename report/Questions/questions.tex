
\documentclass[a4paper, 12pt]{extarticle}
\usepackage[utf8]{inputenc}
\usepackage{amsmath}
\usepackage{amssymb}
\usepackage{bm}
\usepackage{color}
\usepackage{graphicx}
\usepackage{comment}
\usepackage{tikz} 
\usepackage{verbatim}
%\usepackage[a4paper,
%            bindingoffset=0.2in,
%            left=0.5in,
%            right=1in,
%            top=1in,
%            bottom=0.5in,
%            footskip=.25in]{geometry}

\title{IN-STK5000 Project 2 - Project 5}
\author{Anders Bredesen Hatlelid \\
        Jacob Nicolai Arthur Sjødin \\
        \and
        Even Tronstad \\
        Torgeir Ladstein Waagbø 
}

\date{\today}
\begin{document}
\maketitle
\section*{Questions}
\item We have chosen vaccines, and our policy is just to maximise utility. 
The rewards are individual negative constant for each symptom, where we way critical more (less since negative). 
The the utility is the sum of all the constants. The expected utility is then the sum of the product of the rewards and the expectation for each symptoms. 
Our model is beta-Bernoulli bandits. Each action and symptom together make up a bandit. With 10 symptoms and 3 vaccines, we get 40 bandits. 
When individual 100 comes in for vaccines, we give this person a vaccine based on the response symptoms from the first 99 people. 
The problem with this setup is that it does not take into action the features of the individual, we only give a vaccine based on the symptoms registered from the earlier patients. 
This makes it hard 
\end{document}
